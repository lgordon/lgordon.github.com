\begin{center}
{\LARGE Lyle M. Gordon\\[0.3cm]}
\href{mailto:lyle@lylegordon.ca}{lyle@lylegordon.ca}
\end{center}

\section*{Education}
\years{2014}\textbf{PhD}, Materials Science and Engineering\\* 
\emph{Northwestern University}. Evanston, IL.
\begingroup\setlength{\parskip}{0.15cm}

\years{2008}\textbf{Bachelors of Applied Science with Honours}, Materials Science and Engineering\\* 
\emph{University of Toronto}. Toronto, ON.
\endgroup

\section*{Experience}

\years{2017-}\textbf{Senior Scientist}, Nano Precision Medical, Emeryville, CA.\\*
\begingroup\setlength{\parskip}{0.2cm}
Solving materials science challenges to develop a titanium oxide nanotube membrane for a sustained release drug delivery application.

\years{2016-2017}\textbf{Materials Scientist}, Nano Precision Medical, Emeryville, CA.\\*
Developing tools and techniques to characterize structure and chemistry of nanoporous titanium oxide membranes.

\years{2014-2016}\textbf{W.R. Wiley Distinguished Postdoctoral Fellow}, Environmental Molecular Sciences Laboratory\\* Pacific Northwest National Laboratory, Richland, WA.\\*
%\textsc{advisor} : Dr. A. Scott Lea \\*
Understanding the role of nanoporous surfaces on atmospheric ice nucleation using a custom designed low temperature FTIR cell and \emph{in situ} TEM. Member of EMSL microscopy group, responsible for working with internal and external users on materials characterization projects. Experience with in situ SEM, TEM, and atom probe tomography.

\years{2008-2014}\textbf{PhD Candidate}, Biomineral Engineering Group, Materials Science and Engineering\\* Northwestern University, Evanston, IL.\\*
%\textsc{advisor} : Dr. Derk Joester\\*
Applied advanced characterization tools to understand the structure chemistry of biological minerals. Discovered controlling influence of intergranular phases and grain boundary chemistry in tooth enamel on mechanical and chemical properties.

\years{2012-2014}\textbf{Technology Consultant}, PreScouter, Chicago, IL.\\*
Technology scouter connecting corporate innovators to new technologies.  

\years{2007-2008}\textbf{Undergraduate Researcher}, Hybrid Materials Group, Materials Science and Engineering\\* University of Toronto, Toronto, ON.\\*
%\textsc{advisor} : Dr. Glenn D. Hibbard\\*
Designed and fabricated a microscale periodic cellular material using rapid prototyping and electrodeposition of high-strength nanocrystalline nickel. Measured mechanical properties and developed structure-property relationships to optimize composite truss design.

\years{2004-2008}\textbf{Team Captain \& Concrete Mix Lead}, Concrete Canoe Team, Civil Engineering Department\\* University of Toronto, Toronto, ON.\\*
%\textsc{advisor} : Dr. Kim D. Pressnail\\*
Led the development and testing of carbon fiber reinforced lightweight concrete composite. Implemented ASTM standards for mechanical testing. Optimized composite was used to construct a racing canoe.

\years{2007}\textbf{Research Associate}, Advanced Regenerative Tissue Engineering Centre\\* Sunnybrook Health Sciences Centre, Toronto, ON.\\*
%\textsc{advisor} : Dr. Cari M. Whyne \& Dr. Kimberly Woodhouse.\\*
Characterized mechanical properties of a composite hydrogel for tissue engineering of the intervertebral disc. Modeled viscoelastic mechanical response and evaluated cell-material interactions.

\years{2005-2006}\textbf{Research Associate}, Orthopaedic Biomechanics Lab\\* Sunnybrook Health Sciences Centre, Toronto, ON.\\*
%\textsc{advisor} : Dr. Cari M. Whyne.\\*
Began development and experimental validation of a finite element model of pelvic lateral compression fracture stability. Developed a 3D atlas-based method to automate segmentation of metastatic vertebrae on X-ray computed tomography scans. 
\endgroup

%\section*{Skills}
%\textbf{Experimental Techniques:} atom-probe tomography ·  electron microscopy · focused ion beam · infrared spectroscopy · electron energy %loss spectroscopy · X-ray spectroscopy · synchrotron X-ray absorption spectroscopy, diffraction \&  tomography · electron backscatter %diffraction · inductively coupled plasma mass spectroscopy · finite element analysis · nanoindentation · mechanical testing · corrosion %testing.
%\begingroup\setlength{\parskip}{0.1cm}

%\textbf{Programming Languages:} Mathematica · C/C++ · VB.net · Python.
%\endgroup

\section*{Selected Awards}

\years{2014}\textbf{W.R. Wiley Distinguished Postdoctoral Fellowship}, Environmental Molecular Spectroscopy Lab, Pacific Northwest National Lab.
\begingroup\setlength{\parskip}{0.1cm}

\years{2013}\textbf{Microscopy \& Microanalysis Presidential Scholar Award}, Microscopy Society of America and the Microanalysis Society.

\years{2008-2012}\textbf{Postgraduate Scholarship, Masters \& Doctorate}. National Science and Engineering Research Council of Canada.

\years{2009}\textbf{Image of Distinction}. Nikon Small World --- Photomicrography Competition. \href{http://www.nikonsmallworld.com/detail/year/2009/66}{\textsc{\footnotesize{[url]}}}

\years{2008}\textbf{Walter P. Murphy Fellowship}. Materials Science and Engineering, Northwestern University.

\years{2007-2008}\textbf{Stelco Scholarship}. Materials Science and Engineering, University of Toronto.
\endgroup

\section*{Teaching Experience}
\years{2012}\textbf{Guest lecturer}, Biominerals: Hierarchical Architecture and Function, Northwestern University.
\begingroup\setlength{\parskip}{0.15cm}

\years{2009-2011}\textbf{Teaching and laboratory assistant}, Introduction to Materials Science, Northwestern University.\\*
Implemented new discovery-based laboratory experiments and coordinated weekly laboratory sessions.
\endgroup

\section*{Selected Publications}
\years{2016}Nune, S. K., Lao, D., Heldebrant, D. J., Liu, J., Olszta, M. J., Kukkadapu, R., \textbf{Gordon, L.M.}, Nandasiri, M. I., Gotthold, D. W., Schaef, H. T. “Anomalous Water Expulsion from Carbon Rods at High Humidity.” \emph{Nature Nanotechnology} 11, 791.
\begingroup\setlength{\parskip}{0.15cm}

\years{2015}\textbf{Gordon, L.M.}, Cohen, M.J., MacRenaris, K., Pasteris, J.D., Seda, T., Joester, D. “Amorphous Intergranular Phases Control the Properties of Tooth Enamel.” \emph{Science} 347, 6223 (2015).

\years{2015}\textbf{Gordon, L.M.}, Joester, D. “Mapping residual organics and carbonate at grain boundaries and in the amorphous interphase in mouse incisor enamel.” \emph{Frontiers in Physiology} 6, 57. 

\years{2014}Schreiber, D. K., Chiaramonti, A. N., \textbf{Gordon, L.M.}, Kruska, K. “Applicability of post-ionization theory to laser-assisted field evaporation” \emph{Applied Physics Letters} 105, 244106 (2014). 

\years{2014}\textbf{Gordon, L.M.}, Roman, J., Everly, R.M., Cohen, M.J., Wilker, J.J., Joester, D. “Selective Formation of Metastable Ferrihydrite in the Chiton Tooth.” \emph{Angewandte Chemie International Edition} 53, 11506–11509 (2014). 

\years{2012}\textbf{Gordon, L.M.}, Tran, L., Joester, D. “Atom probe tomography of apatites and bone-type mineralized tissues.” \emph{ACS nano} 6, 10667-10675 (2012).

\years{2011}\textbf{Gordon, L.M.}, Joester, D. “Nano-Scale Chemical Tomography of Buried Organic-Inorganic Interfaces in the Chiton Tooth.” \emph{Nature} 469, 194-197 (2011). 

\years{2009}\textbf{Gordon, L.M.}, Bouwhuis, B.A., Suralvo, M., McCrea, J.L., Palumbo, G., Hibbard, G.D. ``Micro-truss nanocrystalline Ni hybrids." \emph{Acta Materialia} 57, 932-939 (2009). 

\years{2009}Leung, A., \textbf{Gordon, L.M.}, Skrinskas, T., Szwedowski, T., Whyne, C.M. ``Effects of bone density alterations on strain patterns in the pelvis: application of a finite element model." \emph{Proceedings of the Institution of Mechanical Engineers, Part H: Journal of Engineering in Medicine} 223, 965-979 (2009). 
\endgroup\\[0.1cm]

\textbf{References and full academic curriculum vitæ available upon request.}
